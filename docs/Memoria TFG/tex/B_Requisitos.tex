\apendice{Especificación de Requisitos}

\section{Introducción}

\section{Objetivos generales}

\section{Catalogo de requisitos}
En este apartado se detallan los requisitos funcionales y no funcionales de la aplicación desarrollada.

\subsection{Requisitos funcionales}
\begin{itemize}
\tightlist
\item 
	\textbf{RF-1 Jugar:} La aplicación debe permitir iniciar una partida.
	
	\begin{itemize}
    \tightlist
    \item
      	\textbf{RF-1.1 Seleccionar modo:} El usuario debe poder escoger entre los modos de juego disponibles.
      	
      	\begin{itemize}
		\tightlist
		\item
			\textbf{RF-1.1.1 Partida contrarreloj:} La aplicación debe poder crear una partida de carrera a contrarreloj.
    	\end{itemize}
    	\item
    		\textbf{RF-1.2 Seleccionar personaje:} El usuario debe poder escoger entre las diferentes personajes disponibles para jugar.
		\item    	
    		\textbf{RF-1.3 Seleccionar circuito:} El usuario debe poder escoger entre los diferentes escenarios disponibles en los que realizar la carrera.
    	
    \end{itemize}
    
	\item \textbf{RF-2 Créditos:} La aplicación ha de ofrecer la posibilidad de conocer al usuario quiénes han realizado la aplicación en los distintos aspectos que la componen.
	\item \textbf{RF-X :}
\end{itemize}
\subsection{Requisitos no funcionales}
\begin{itemize}
\tightlist
\item 
	\textbf{RNF-1 Usabilidad:} La aplicación debe ser sencilla de jugar y divertida. La curva de aprendizaje ha de ser baja para que el usuario no desista en comprender el funcionamiento y mejorar, a la vez que tiene que ser capaz de divertirse gracias a las mecánicas del videojuego. Por ello, la interfaz es \"amigable\" de cara al usuario.

\item 
	\textbf{RNF-2 Rendimiento:} La aplicación debe poder ser ejecutada en computadoras de gama media, con tiempos de carga entre pantallas aceptables y una jugabilidad fluida sin ralentizaciones de \textit{frames}.
	
\item 
	\textbf{RNF-3 Escalabilidad:} La aplicación ha de permitir la adición de nuevas funcionalidades y mejoras de forma relativamente sencilla.
	
\item 
	\textbf{RNF-4 Disponibilidad:} La aplicación debe estar disponible siempre para su uso, independientemente de si se dispone o no de conexión a internet u otros programas (que no sean \textit{drivers} o librerías gráficas necesarias para la ejecución del videojuego).
	
\item 
	\textbf{RNF-5 Mantenibilidad:} La aplicación debe permitir recibir cambios y correcciones de fallos que afecten al rendimiento del programa.
	
\item 
	\textbf{RNF-6 Soporte:} La aplicación ha de poder ser usada en nuevas versiones de sistemas operativos.

\item 
	\textbf{RNF-X Internacionalización:} 
\end{itemize}
\section{Especificación de requisitos}


