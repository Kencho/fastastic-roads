	\capitulo{3}{Conceptos teóricos}

\section{Desarrollo de videojuegos}

El desarrollo de videojuegos es el proceso a seguir para la creación de un videojuego. En este proceso van incluidos multitud de disciplinas, pues abarca la programación, el diseño gráfico y la animación, el sonido y la música, entre otros, además de la parte que nos toca como es la programación.

\subsection{Narrativa}

Puesto que se trata de un videojuego de carreras, la narrativa está prácticamente ausente, no careciendo de importancia de igual manera, pues lo que se pretende con este juego a posteriori es el mostrar históricamente la implicación de las personajes en los vehículos que manejan.

\subsection{Arte}

Para la parte artística, el resto del equipo ha sido el encargado de crear los modelos 3D para los vehículos, escenarios y animaciones bajo el software de código abierto Blender. 

Inicialmente, en el mismo motor Unity se creó un escenario de pruebas simple, cogiendo dos objetos de tipo cubo aplanados y estirados para hacer superficie sobre superficie. De esta manera, obteníamos un pequeño desnivel por el que, a continuación, bajo otro modelo de vehículo simple hecho a partir de cilindros como cuerpo y ruedas, realizábamos las pruebas.

Más adelante, los compañeros me adjuntaron un escenario de pruebas similar al final en estructura, de manera que ahí podría ir haciendo pruebas con el vehículo, respecto a su configuración, así como todo lo relacionado con el HUD (conteo de vueltas, tiempo, cuenta atrás, ...) y otras características.

\subsection{Programación}

desarrollo y programación (fases, etc.)

\section{Motor de videojuegos}

El corazón de un videojuego es el motor. Como definición, se trata de una serie de rutinas de programación que forman el núcleo de un juego, permitiendo el diseño, la creación y el funcionamiento del mismo. También se llama motor de videojuegos al software que nos permite crearlos, del cual hemos hecho uso para desarrollar Fastastic Roads.

Explicar cómo funcionan, peculiaridades, cuáles hay y por qué escojo Unity

\subsubsection{Proyección ortográfica}

En este tipo de proyección, el tamaño no indica la profundidad. Los objetos no se ven más pequeños si se alejan y las líneas paralelas en el espacio se renderizan como paralelas.

\subsubsection{Proyección en perspectiva}

Las líneas paralelas en el espacio parecen tocarse en el infinito. Es la representación al “mundo real”.

\subsubsection{Propiedades de la transformación}

Las propiedades de la transformación se realizan en espacio global, es decir, son relativas a la escena en la que nos hallemos. En caso de que un objeto tenga un padre, estas propiedades serán en espacio local, lo que implica que su posición será relativa a la posición del padre y no de la escena en sí.

\section{C\#}

Peculiaridades y su uso en Unity.

\section{Videojuegos de carreras} 

Físicas, colliders, triggers, cosas que son importantes y que no hay en otros videojuegos, etc.