	\capitulo{3}{Conceptos teóricos}
El vehículo modelo de prueba se compone de: 
\\- Cuatro ruedas (cuatro modelos) que contienen un WheelCollider cada uno (cuatro colliders en total) por separado. Los cuatro modelos de ruedas van en un grupo aparte llamado “Models” y los colliders en otro llamado “Colliders”.
\\- Un centro de masa para indicar el centro de gravedad del vehículo, el cual irá en el grupo de “Colliders”.
\\- La suspensión del vehículo va en otro módulo aparte vacío.
\\\\
\\\\
En aquellos proyectos que necesiten para su comprensión y desarrollo de unos conceptos teóricos de una determinada materia o de un determinado dominio de conocimiento, debe existir un apartado que sintetice dichos conceptos.

Algunos conceptos teóricos de \LaTeX \footnote{Créditos a los proyectos de Álvaro López Cantero: Configurador de Presupuestos y Roberto Izquierdo Amo: PLQuiz}.

\section{Proyección ortográfica}

En este tipo de proyección, el tamaño no indica la profundidad. Los objetos no se ven más pequeños si se alejan y las líneas paralelas en el espacio se renderizan como paralelas.

\subsection{Subsecciones}

Además de secciones tenemos subsecciones.


\subsubsection{Subsubsecciones}

Y subsubsecciones. 


\section{Proyección en perspectiva}

Las líneas paralelas en el espacio parecen tocarse en el infinito. Es la representación al “mundo real”.

\section{Propiedades de la transformación}
Las propiedades de la transformación se realizan en espacio global, es decir, son relativas a la escena en la que nos hallemos. En caso de que un objeto tenga un padre, estas propiedades serán en espacio local, lo que implica que su posición será relativa a la posición del padre y no de la escena en sí.

\section{Imágenes}

Se pueden incluir imágenes con los comandos standard de \LaTeX, pero esta plantilla dispone de comandos propios como por ejemplo el siguiente:

\imagen{escudoInfor}{Autómata para una expresión vacía}



\section{Listas de items}

Existen tres posibilidades:

\begin{itemize}
	\item primer item.
	\item segundo item.
\end{itemize}

\begin{enumerate}
	\item primer item.
	\item segundo item.
\end{enumerate}

\begin{description}
	\item[Primer item] más información sobre el primer item.
	\item[Segundo item] más información sobre el segundo item.
\end{description}
	
\begin{itemize}
\item 
\end{itemize}

\section{Tablas}

Igualmente se pueden usar los comandos específicos de \LaTeX o bien usar alguno de los comandos de la plantilla.

\tablaSmall{Herramientas y tecnologías utilizadas en cada parte del proyecto}{l c c c c}{herramientasportipodeuso}
{ \multicolumn{1}{l}{Herramientas} & App AngularJS & API REST & BD & Memoria \\}{ 
HTML5 & X & & &\\
CSS3 & X & & &\\
BOOTSTRAP & X & & &\\
JavaScript & X & & &\\
AngularJS & X & & &\\
Bower & X & & &\\
PHP & & X & &\\
Karma + Jasmine & X & & &\\
Slim framework & & X & &\\
Idiorm & & X & &\\
Composer & & X & &\\
JSON & X & X & &\\
PhpStorm & X & X & &\\
MySQL & & & X &\\
PhpMyAdmin & & & X &\\
Git + BitBucket & X & X & X & X\\
Mik\TeX{} & & & & X\\
\TeX{}Maker & & & & X\\
Astah & & & & X\\
Balsamiq Mockups & X & & &\\
VersionOne & X & X & X & X\\
} 
