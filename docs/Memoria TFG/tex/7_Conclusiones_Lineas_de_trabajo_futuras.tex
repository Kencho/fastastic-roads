\capitulo{7}{Conclusiones y Líneas de trabajo futuras}

Para finalizar, en este capítulo se exponen las conclusiones extraídas del proyecto, así como las líneas de trabajo a seguir en el futuro.

\section{Conclusiones}

Tras haber finalizado el desarrollo del proyecto, se ha podido comprobar cómo se han cumplido los objetivos del proyecto de manera satisfactoria. 

Como se ha mencionado anteriormente, la realización de un videojuego requiere de varias personas distribuidas por los diferentes campos (programación, modelaje, etc.), donde se ha tenido la oportunidad de poder contar con un equipo dedicado a la parte artística, para así dedicarse de manera individual a la parte relacionada con la programación y el \textit{software}. 

Por ello, es un proyecto con bastante ambición, donde se calcularon una serie de requisitos que sí podían llegarse a completar con este número reducido de personas encargadas, y el resultado ha sido altamente satisfactorio.

Se ha partido del desconocimiento absoluto del motor, requiriendo de su aprendizaje en funcionalidades, manejo, gestión de archivos, configuración por proyecto, programación y otros aspectos, consiguiendo en pocos meses los conocimientos básicos para poder completar este juego y con un gran camino por delante para seguir aprendiendo.

Así mismo, se ha conseguido realizar los propósitos planteados como videojuego de carreras contrarreloj, con menús amigables y comprensibles tanto para aquellas personas que no estén familiarizadas como para las que sí y con una ausencia mayoritaria de errores en los componentes esenciales.

Y no solo se ha obtenido un juego de carreras contrarreloj jugable, también disfrutable, pudiendo completar vueltas en un entorno virtual agradable, invitando al usuario a tratar de batir su propio récord empezando tantas partidas como desee, hasta que voluntariamente decida dejar de jugar.

Por supuesto, se ha comprendido el costoso proceso que conlleva realizar un proyecto de esta magnitud, concretamente un proyecto relacionado con videojuegos. Muchos elementos básicos que poseen los videojuegos generalmente y se presuponen que son intrínsecos de base no son triviales.

En definitiva, ha sido una oportunidad para aprender y comprender el ámbito de los videojuegos y todo lo que le rodea, sentando unas bases para poder continuar otros proyectos similares relacionados con videojuegos o diseño y programación en Unity.

\section{Líneas de trabajo futuras}

Este proyecto no ha llegado a su fin, pues ``Fastastic Roads'' es un videojuego con intención de continuar su desarrollo en ÍTACA. Por ello, los objetivos a futuro del proyecto son los siguientes:
\begin{itemize}
\tightlist
    \item Integrar nuevos circuitos.
    \item Integrar los nuevos modelos de vehículos para escoger.
    \item Crear un comportamiento determinado para cada vehículo, implementando habilidades que hagan a cada uno de ellos único y diferente del resto.
    \item Investigar la creación de jugadores no humanos (CPU) para competir en las partidas con ellos e integrar ese sistema en el videojuego.
    \item Crear un modo de carrera competitiva. Este modo permitirá competir con otros jugadores no humanos, donde el ganador es el primero en llegar a meta. Para hacer el modo más atractivo, los vehículos podrán recoger determinados ítems que estarán en la escena colocados para su uso contra los rivales.
    \item Crear un modo multijugador a pantalla partida, permitiendo al usuario competir con otra persona en un mismo ordenador en los distintos modos que ofrece el juego.
    \item Mejorar la interfaz gráfica para que sea más agradable y llamativo para el usuario.
    \item Integrar un sistema de audio, así como los respectivos sonidos de efectos y música al juego.
    \item Hacerlo más accesible, dando la posibilidad de configurar distintas opciones como son las opciones de volumen, pantalla y controles, entre otros.
    \item Nuevas funcionalidades, mejoras y correcciones que vayan surgiendo durante el transcurso del tiempo.
\end{itemize}