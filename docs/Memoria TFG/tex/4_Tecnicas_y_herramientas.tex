\capitulo{4}{Técnicas y herramientas}
Ante la variedad de motores de videojuegos, hubo que escoger uno entre todos los disponibles. Puesto que cada uno tiene sus virtudes y sus complejidades a la hora de realizar determinadas acciones, se decidió analizar tres de los principales usados en el mercado global de videojuegos y audiovisuales, los cuales eran CryEngine, Unity y Unreal Engine. La mayor ventaja que se le sacó al escogido finalmente, Unity, fue el conocimiento previo del lenguaje de programación en el que se realizan los scripts, el cual es C\#, a diferencia del que usan los otros dos motores, que es C++. Partiendo de esta premisa, se comenzó a trabajar en el proyecto.

\section{Unity}

Unity 

\section{Lenguaje de programación C\#}

Peculiaridades y su uso en Unity.
Programación en Visual Studio

\section{LaTeX}

\section{Metodologías}

\section{Control de versiones}
GitHub

\section{Patrones de diseño}

Esta parte de la memoria tiene como objetivo presentar las técnicas metodológicas y las herramientas de desarrollo que se han utilizado para llevar a cabo el proyecto. Si se han estudiado diferentes alternativas de metodologías, herramientas, bibliotecas se puede hacer un resumen de los aspectos más destacados de cada alternativa, incluyendo comparativas entre las distintas opciones y una justificación de las elecciones realizadas. 
No se pretende que este apartado se convierta en un capítulo de un libro dedicado a cada una de las alternativas, sino comentar los aspectos más destacados de cada opción, con un repaso somero a los fundamentos esenciales y referencias bibliográficas para que el lector pueda ampliar su conocimiento sobre el tema.