\capitulo{6}{Trabajos relacionados}

El número de videojuegos disponibles al público es ingente, incluso considerando únicamente aquellos que se han creado en Unity, debido a la disponibilidad global del motor así como a la cantidad de juegos que se crean mensual y diariamente.

Es por ello que los trabajos relacionados que se mencionan a continuación tienen en común el género de carreras al cual pertenecen, entre ellos un Trabajo de Fin de Grado desarrollado en Unity y realizado en la Universidad de Burgos.

\section{Mario Kart}

\textit{Mario Kart} es un videojuego de carreras de karts y a la vez una serie de videojuegos con el mismo nombre desarrollados por la compañía japonesa Nintendo. Es mundialmente conocido, además de por los personajes pertenecientes a la saga ``Super Mario'' como por su fácil accesibilidad para jugadores no experimentados, así como para aquellos que tengan más habilidad con este género de videojuegos.

Las mecánicas de este videojuego son una inspiración y modelo a seguir para ``Fastastic Roads'', por el subgénero de carreras al que pertenecen ambos y por sus modos de juego, ya que contiene tanto el modo a contrarreloj, el cual ha sido implementado en este proyecto, como el modo Grand Prix, en el cual se compite con los distintos contrincantes para llegar antes a la meta haciendo uso de distintos objetos distribuidos por el escenario que ofrecen ventajas al jugador, tanto para ``atacar'' al resto de jugadores como para beneficiar temporalmente al mismo, siendo este modo un objetivo a implementar en ``Fastastic Roads''. Aparte incluye otros dos modos, un ``versus'' (todos contra todos en carreras) y uno de batalla múltiple (equipo contra equipo con objetivos).

Además, cada vehículo tiene mejores o peores habilidades respecto a velocidad, aceleración, peso, manejo y agarre.

\section{Crash Team Racing}

Se trata de un videojuego de karts muy inspirado en ``Mario Kart''. Contiene sus mismos modos de juego, así como alguno nuevo añadido, pero siguiendo en general las mismas líneas de jugabilidad, con personajes relacionados con los videojuegos de \textit{Crash Bandicoot}, vehículos con mejoras en velocidad, aceleración y giro, etc.

\section{UBURACE}

Este proyecto, creado por Uxía Doval Blanco en el año 2016 para el ``Grado en Comunicación Audiovisual'', es un videojuego de carreras creado con la intencionalidad de aprender a realizar un juego cumpliendo las indicaciones de un cliente con todos los pasos que conlleva. Está planteado como prototipo desde el ámbito de la Comunicación Audiovisual, siendo la parte de desarrollo de \textit{software} secundaria y, por lo tanto, poco relevante para este estudio. 

Este proyecto era ambicioso respecto a que una misma persona era la encargada de todas las fases: ilustración, modelado y texturizado 3D, animación y programación. 

Los requisitos para la realización de este proyecto de juego de carreras son un contador de vueltas y un sumador de puntos en base a ciertos objetos disponibles. Los modelos tienen \textit{colliders} sencillos y el funcionamiento es más simple, aprovechando en el apartado de programación \textit{assets} previamente programados sin seguir ningún tipo diseño de \textit{software}. Contiene dos escenarios, estando uno de ellos basado en la Facultad de Económicas de la Universidad de Burgos. Al tratarse de un proyecto de Comunicación Audiovisual, está más enfocado a ese campo junto a los principios de diseño de videojuegos que a ser un proyecto enfocado al \textit{software}.

