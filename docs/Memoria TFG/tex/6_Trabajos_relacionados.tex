\capitulo{6}{Trabajos relacionados}

El número de videojuegos disponibles al público es ingente, incluido aquellos que solo se han creado en Unity, debido a la disponibilidad global del motor así como a la cantidad de juegos que se crean mensual y diariamente.

Es por ello que los trabajos relacionados que se mencionan a continuación tienen en común dos aspectos: han sido desarrollados en Unity y han sido Trabajos de Fin de Grado realizados en la Universidad de Burgos. Estos dos videojuegos están planteados como prototipos desde el ámbito de la Comunicación Audiovisual, siendo la parte de desarrollo de \textit{software} secundaria y, por lo tanto, poco relevante para este estudio.

\section{UBURACE}

Este proyecto, creado por Uxía Doval Blanco en el año 2016 para el ``Grado en Comunicación Audiovisual'', es un videojuego de carreras creado con la intencionalidad de aprender a realizar un juego cumpliendo las indicaciones de un cliente con todos los pasos que conlleva. Había dos escenarios, estando uno de ellos basado en la Facultad de Económicas de la Universidad de Burgos, y tenía 

Este proyecto era ambicioso respecto a que una misma persona era la encargada de todas las fases: ilustración, modelado y texturizado 3D, animación y programación. 

Los requisitos para la realización de este proyecto de juego de carreras eran un contador de vueltas y un sumador de puntos en base a ciertos objetos disponibles. Los modelos tenían \textit{colliders} sencillos y el funcionamiento era más simple, aprovechando en el apartado de programación \textit{assets} previamente programados sin seguir ningún tipo diseño de \textit{software}. Al tratarse de un proyecto de Comunicación Audiovisual, iba más enfocado a ese campo junto a los principios de diseño de videojuegos que a ser un proyecto enfocado al \textit{software}.

\section{Error}

``Error'' es la primera iniciativa desde el Área de Comunicación Audiovisual de crear un videojuego (ya que desde el Grado en Ingeniería Informática hubo algún proyecto años atrás, como ``Genen'') creado por Javier Mateos en el año 2014.

Estaba fundamentalmente basada en la narrativa y en el modelado 3D. En la narrativa la idea es la descomposición del juego, donde empieza siendo un RPG medieval pero a medida que avanza se va ``rompiendo'' todo: aparecen elementos que no deberían, se puede viajar entre niveles en 3D y, al final, acaba fallando por completo todo el juego y se llega a la placa base (de ahí el nombre del mismo). 

Se trata de un juego puramente de narrativa, de tipo \textit{walking simulator} (estilo de juego en el que, por lo general, hay que caminar para descubrir la historia que esconde, mediante mecánicas de búsqueda de expresividad e innovación sin apenas interactividad) y en el la programación también ha sido adaptada en base a componentes previos existentes (en este caso en \textit{JavaScript}, pues era el lenguaje que años atrás manejaba Unity). 