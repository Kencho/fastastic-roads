\capitulo{1}{Introducción}
Los videojuegos son una forma de entretenimiento extendida a nivel global. La riqueza en variedad de géneros, estilos a nivel gráfico y narrativo, bandas sonoras, animaciones y demás aspectos artísticos es muy amplia, por lo que nos permite experimentar creando juegos casuales, serios, simuladores y otros más experimentales.

Según la Asociación Española de Videojuegos (AEVI), el mercado internacional del videojuego creció un 9,6\% en el año 2019 con respecto al año anterior, alcanzando una facturación total de 133.670 millones de euros[aevi:internacional]. A nivel estatal, según el primer informe económico en el que se analizó el impacto de la industria de los videojuegos sobre la contabilidad nacional realizado en enero de 2018, la industria de los videojuegos equivale al 0,11\% del PIB español, donde por cada euro invertido en este sector se tiene un impacto de 3 euros, dando un total de 3577 millones de euros y aproximadamente 22.800 empleos a fecha 2016[aevi:nacional]. 

Teniendo en cuenta estos datos, se es consciente del alcance que tienen los videojuegos y de la oportunidad que puede ser aprovechar los recursos disponibles para crear uno que pueda atraer a las personas y tenga un sentido didáctico. Esta idea hace unos años era concebible, pero muy difícilmente realizable, no solo por los escasos recursos que podía haber en Internet o en los libros, sino porque las herramientas no estaban disponibles para un usuario estándar que no pudiese invertir en programas con licencias de pago, o en su defecto no estaban lo suficientemente desarrolladas.

Por suerte, gracias, sobre todo, al avance tecnológico, hoy en día se dispone de herramientas muy potentes que nos ofrecen un ecosistema preparado para poder crear y desarrollar videojuegos al alcance de cualquier persona que tenga un ordenador que cumpla los requisitos mínimos para un correcto funcionamiento y una conexión a internet para poder descargarlos.

En este proyecto se ha partido desde un motor escogido en base al análisis general de los disponibles, sin conocimiento sobre el funcionamiento del mismo, hasta tratar de conseguir un videojuego jugable, disfrutable y accesible al usuario, así como trabajado en un entorno multidisciplinar relacionado con los videojuegos. Con todo ello, se pretende principalmente comprender la complejidad de un proyecto de esta magnitud, cómo funcionan los motores de videojuegos, cómo se pueden solventar los problemas al respecto y, sobre todo, disfrutar del trayecto y del resultado obtenido.

\section{Materiales adjuntos}

Los materiales del proyecto adjuntados son los siguientes:

\begin{itemize}
\tightlist
	\item Aplicación para Windows ``Fastastic Roads''.
	\item Proyecto creado en Unity para su apertura en el motor.
	\item Vídeo explicativo dentro del videojuego.
\end{itemize}