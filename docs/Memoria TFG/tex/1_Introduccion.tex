\capitulo{1}{Introducción}
Los videojuegos son una forma de entretenimiento extendida a nivel global. La riqueza en variedad de géneros, estilos a nivel gráfico y narrativo, bandas sonoras, animaciones y demás aspectos artísticos es muy amplia, por lo que nos permite experimentar creando juegos casuales, serios, simuladores y otros más experimentales.

Según la Asociación Española de Videojuegos (AEVI), el mercado internacional del videojuego creció un 9,6\% en el año 2019 con respecto al año anterior, alcanzando una facturación total de 133.670 millones de euros[aevi:internacional]. A nivel estatal, según el primer informe económico en el que se analizó el impacto de la industria de los videojuegos sobre la contabilidad nacional realizado en enero de 2018, la industria de los videojuegos equivale al 0,11\% del PIB español, donde por cada euro invertido en este sector se tiene un impacto de 3 euros, dando un total de 3577 millones de euros y aproximadamente 22.800 empleos a fecha 2016[aevi:nacional]. 



Descripción del contenido del trabajo y del estructura de la memoria y del resto de materiales entregados.
