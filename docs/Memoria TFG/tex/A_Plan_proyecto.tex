\apendice{Plan de Proyecto Software}

\section{Introducción}
En todo proyecto se requiere de una planificación de cara a cumplir plazos, presupuestos, obligaciones legales, etc. Es por ello que lo ejemplar es seguir una metodología de planificación que se adecue al proyecto, y como viene dándose en los últimos años en el mundo del software, en general, suele ser apropiado seguir la metodología SCRUM gracias a su capacidad incremental y la comunicación entre las personas implicadas.
\\\\
Este apartado contiene la explicación de la planificación del proyecto, así como el estudio de viabilidad del mismo (económica, calculando costes que pueden conllevar, y legal, teniendo en cuenta la legislación vigente).

\section{Planificación temporal}
Aunque este proyecto no ha sido realizado en conjunto a más personas pertenecientes a un mismo equipo, sino que se trata de un proyecto académico individual, sin reuniones diarias y más puntos clave dentro de esta metodología, se han seguido varias pautas dentro de la metodología SCRUM, como son la creación de Sprints de, aproximadamente, dos semanas de duración y el \textit{feedback} rápido. 
\\\\
Se ha hecho uso de GitHub de forma exclusiva, 
\section{Estudio de viabilidad}

\subsection{Viabilidad económica}

\subsection{Viabilidad legal}


