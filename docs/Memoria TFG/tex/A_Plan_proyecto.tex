\apendice{Plan de Proyecto Software}

\section{Introducción}

En todo proyecto se requiere de una planificación debidamente organizada, con el fin de cumplir plazos, presupuestos, obligaciones legales y otros aspectos de manera adecuada. 

Es por ello que lo ideal es seguir una metodología de planificación adaptada al proyecto, con la que se puedan aprovechar de la mejor forma posible los recursos disponibles en el tiempo marcado de cara a la correcta finalización, tanto de este Trabajo de Fin de Grado como de cualquier otro tipo de proyecto.

\section{Planificación temporal}

Inicialmente, se plantearon distintas metodologías de desarrollo \textit{software}, entre ellas Kanban y SCRUM.

Como se explica en el capítulo de ``Técnicas y Herramientas'' en el documento de memoria, Kanban permite un trabajo más dinámico y fluido, con un proceso evolutivo e incremental enfocado al desarrollo de tareas pendientes. Es más adecuado para los procesos de integración continua, en las que no se tiene periódicamente entregas o microentregas que hacer y se definen una serie de tareas para esas entregas, sino que se van tomando esas tareas, luego en función de su prioridad se van haciendo y finalmente se van validando tan pronto como estén antes de pasar a la siguiente tarea.

Debido a la magnitud del proyecto, se necesitaron ciclos mucho más rápidos de revisión de las tareas para poder ir avanzando y que el impacto de cualquier ``cuello de botella'' fuese el menor posible.

Mediante un tablero de proyecto con cuatro columnas, correspondientes a \textit{``To do''} o ``Por hacer'', \textit{``In progress''} o ``En progreso'', \textit{``QA/Testing''} o ``Control de calidad'' y \textit{``Done''} o ``Hecho'', el flujo de trabajo ha funcionado de la siguiente manera:
\begin{itemize}
\tightlist
    \item En la columna de \textbf{\textit{``To do''}} se han ido creando las tareas a medida que se iban necesitando realizarlas. Muchos elementos en el proyecto no son necesarios hasta que se realizan otras tareas previas, a la vez que pueden ir surgiendo a medida que van necesitándose en el momento o acaban siendo pensadas. La tarea correspondiente tendrá una descripción, una duración estimada de tiempo que puede conllevar realizarla y una duración real en la que se marca el tiempo que se ha tardado en realizarla.
    \item En \textbf{\textit{``In progress''}} se tienen todas las tareas que se están realizando. Puede haber varias a la vez, pues se pueden compaginar tengan correlación entre ellas o no, aunque lo ideal es dedicarse a unas pocas que, al menos, dependan entre ellas para no tener una carga de trabajo que no se pueda abarcar. Las tareas irán asignadas a uno o varios usuarios que estén realizándolas.
    \item En la columna de \textbf{\textit{``QA/Testing''}} se tienen todas las tareas tentativamente completadas y listas para su validación. El usuario habrá marcado el tiempo que le ha llevado realmente realizarla. Si cumple el objetivo de la tarea correctamente, se da por finalizada. En caso de tener fallos o de requerir cambios, se devuelve a ``En progreso'', devolviendo la asignación al usuario o usuarios encargados de la misma.
    \item Finalmente, en \textbf{\textit{``Done''}} se tienen todas las tareas dadas por finalizadas. El control de calidad habrá decidido que está correctamente terminada.
\end{itemize}

Una idea fundamental de Kanban es medir los tiempos. En este proyecto, aunque no se ha hecho uso de SCRUM, se han empleado \textit{sprints} de dos semanas de duración para hacer una división de tareas por tiempo, calculando el tiempo empleado en cada una de ellas. 

Con estos datos, por una parte, puede ser un punto de referencia para ajustar en el futuro las estimaciones que se hagan, y por otra para aquellas personas que realicen hacer proyectos similares, al haber utilizado esta taxonomía de tareas, pueden tomar este histórico de valores como referencia para el coste de cuánto supondría una tarea referida a ese tipo de tareas (si son tareas de mejora, calcular cuánto se podría tardar).

\imagen{issues-ratio}{Gráfico del ratio de horas estimadas vs. horas reales empleadas.}

El tiempo estimado en total para las tareas ha sido de aproximadamente 170 horas, y las horas reales empleadas han sido aproximadamente 267. El ratio por tareas se puede comprobar en la gráfica previa. Esta gran diferencia entre las distintas tareas, así como entre la duración calculada total y la real total, viene dada por la magnitud del proyecto y la inexperiencia al respecto. Un videojuego se compone de una gran cantidad de elementos y funciones que no siempre son fácilmente implementables, y la gran mayoría de veces que puede aparentar ser sencillo acaba llevando más tiempo por la investigación adicional correspondiente, la corrección de fallos, la posterior optimización, etc. Es por ello que Kanban ha sido la metodología de planificación ideal, permitiendo esa flexibilidad ante cálculos incorrectos y otros imprevistos. De igual manera, el tiempo total estimado y el tiempo total medido no son realistas, porque en las primeras etapas no se incluyeron las tareas de investigación y aprendizaje, y cuando se empezaron a incluir, no se estimaron inicialmente, de manera que no han quedado reflejadas en ninguna parte.

Previa a la planificación, hubo una gran cantidad de tiempo dedicada al aprendizaje respecto al uso de Unity, sus funcionalidades y otros aspectos, como se indicó en la memoria. La mayor parte de las tareas del proyecto fueron dirigidas a la implementación de las nuevas funciones, así como a la corrección, mejora y optimización de las mismas.

\section{Estudio de viabilidad}

En todo proyecto se requiere de un estudio de viabilidad para asegurar su realización completa. Es necesario analizar su viabilidad económica, con los costes del proyecto que puede conllevar, así como su viabilidad legal, haciendo mención a las licencias que pueda tener.

\subsection{Viabilidad económica}

Se dividen entre costes del proyecto y beneficios obtenidos.

\subsubsection{Costes}

Los costes son aquellos gastos que conlleva realizar algunas actividades, así como la adquisición de materiales y otros aspectos. Se divide en costes fijos y costes directos e indirectos \cite{costes}.

\textbf{\textit{Costes fijos}}

Los costes fijos son aquellos costes que no cambian en función del volumen de actividad, i.e., aquellos que no varían ajenos al proyecto, siendo el más destacable el de los salarios. Estos datos son calculados en base a 4 meses de trabajo.

\begin{table}[H]
	\centering
	\begin{tabular}{>{\raggedright}b{0.6\linewidth}>{\raggedleft\arraybackslash}b{0.2\linewidth}}
		\toprule
		\textbf{{Concepto}}  & \textbf{{Coste}} \\
		\midrule
		\small{\; Contrato de Luz} & \small{180,00 €} \\		
		\midrule
		\small{\; Servicio de Internet} & \small{200,00 €} \\		
		\midrule
		\small{\; Salario x 4 meses} & \small{9251,12 €} \\
		\cline{2-2}
		\scriptsize{\qquad Salario mensual bruto} & \scriptsize{2312,78 €} \\
		\scriptsize{\qquad Retenciones por IRPF (24 \%)} & \scriptsize{252,00 €} \\
		\scriptsize{\qquad Cuotas a la Seguridad Social (28,30 \%)} & \scriptsize{321,30 €} \\
		\scriptsize{\qquad Salario mensual neto} & \scriptsize{1050,00 €} \\
		\bottomrule
		\textbf{{\scriptsize TOTAL}}  & \textbf{\small 9631,12 €} \\
		\bottomrule
	\end{tabular}
	\caption{Tabla de costes fijos.}
\end{table}

\textbf{\textit{Costes directos}}

Los costes directos son los que vienen derivados del desarrollo del proyecto (contratación de servicios para el alojamiento del proyecto, determinado \textit{software} necesario para el mismo, \textit{marketing} y otros).

\begin{table}[H]
	\centering
	\begin{tabular}{>{\raggedright}b{0.6\linewidth}>{\raggedleft\arraybackslash}b{0.2\linewidth}}
		\toprule
		\textbf{{Concepto}}  & \textbf{{Coste}} \\
		\midrule
		\small{\scriptsize Impresión de Memoria} & \small{30,00 €} \\
		\bottomrule
		\textbf{{\scriptsize TOTAL}}  & \textbf{\small 30,00 €} \\
		\bottomrule
	\end{tabular}
	\caption{Tabla de costes directos.}
\end{table}

\textbf{\textit{Costes indirectos}}

Los costes indirectos son aquellos que no dependen del desarrollo del proyecto de manera directa (licencias \textit{software} adquiridas, componentes \textit{hardware} y otros). El cálculo de los costes amortizados se ha realizado tomando como base los 4 meses de trabajo mencionados atrás a lo largo de un plazo de 4 años.

\begin{table}[H]
	\centering
	\begin{tabular}{>{\raggedright}m{0.6\linewidth}>{\raggedleft}m{0.16\linewidth}>{\raggedleft\arraybackslash}m{0.15\linewidth}}
		\toprule
		\textbf{{Concepto}}  & \textbf{{Coste}} & \textbf{{Coste amortizado}} \\
		\midrule
		\small{\scriptsize Ordenador personal} & \small{1999,00 €} & \small{166.58 €} \\
		\small{\scriptsize Licencia Windows 10 Home} & \small{145,00 €} & \small{12,08 €} \\
		\bottomrule
		\textbf{{\scriptsize TOTAL}}  & \textbf{\small 2144,00 €} & \textbf{\small 178,66 €} \\
		\bottomrule
	\end{tabular}
	\caption{Tabla de costes indirectos.}
\end{table}

\subsubsection{Beneficios}

Al tratarse de un proyecto de carácter educativo sin un objetivo de mercado, no se obtiene ningún beneficio económico de explotación. No obstante, tantea ciertos escenarios que, en el futuro con nuevos proyectos, será necesario explorarlos

\subsection{Viabilidad legal}

Para poder distribuir el \textit{software}, se tienen que cumplir determinadas obligaciones legales. La más importante tiene que ver con las licencias, por lo que las nombraremos a continuación.

\subsubsection{Licencias}

El proyecto no tiene una finalidad comercial. No obstante, esto no lo despoja de derechos de autor y distribución. Es por ello que la licencia de distribución escogida ha sido ``Creative Commons'', concretamente la licencia de ``Reconocimiento - No Comercial - Sin Obra Derivada 2.5 España'' (CC BY-NC-ND 2.5 ES)\footnote{Consultable en: \url{https://creativecommons.org/licenses/by-nc-nd/2.5/es/}}. Esta licencia permite compartir la obra en cualquier medio o formato bajo el reconocimiento de la autoría del producto, sin que se pueda utilizar para una finalidad comercial y sin posibilidad de difundir material remezclado o transformado a partir de este.

Respecto al resto de \textit{software} utilizado, estas son sus correspondientes licencias:

\begin{table}[H]
	\centering
	\begin{tabular}{>{\raggedright}b{0.28\linewidth}>{\raggedright}b{0.3\linewidth}>{\raggedleft\arraybackslash}b{0.25\linewidth}}
		\toprule
		\textbf{Herramienta} & \textbf{Descripción} & \textbf{Licencia} \\
		\midrule
		\small{\textbf{Unity}} & \small{Motor de videojuegos.} & \small{Gratuito/Privativo} \\
		\midrule
		\small{\textbf{Blender}} & \small{\textit{Software} de modelaje.} & \small{GPL} \\
		\midrule
		\small{\textbf{Texmaker}} & \small{Editor multiplataforma de texto.} & \small{GPL 2} \\
		\midrule
		\small{\textbf{Git}} & \small{Repositorio de código.} & \small{GPL 2} \\
		\midrule
		\small{\textbf{GitHub}} & \small{Repositorio de código.} & \small{Propietaria/Privativa} \\
		\midrule
		\small{\textbf{StarUML}} & \small{Herramienta de diseño UML.} & \small{Propietaria/Privativa} \\
		\bottomrule
	\end{tabular}
	\caption{Herramientas con sus respectivas licencias.}
\end{table}